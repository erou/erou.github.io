\documentclass[a4paper,12pt,twocolumn]{article}
\usepackage[utf8]{inputenc}
\usepackage[T1]{fontenc}
\usepackage[french]{babel}
\usepackage{amsmath,amssymb,amsthm,amsopn}
\usepackage{mathrsfs}
\usepackage{graphicx}
\usepackage{hyperref}
%\usepackage{tikz}
%\usepackage{array}
\usepackage[landscape]{geometry}
%\usepackage{listings}
%\usepackage{xcolor}
\usepackage{fancyhdr}
\pagestyle{fancy}
\fancyhead[L]{BCPST 2 - Lycée Jacques Prévert}
\fancyhead[R]{Intégrales impropres}
\pagenumbering{gobble}

% Création des labels Théorème, Lemme, etc...

\newtheoremstyle{break}%
{}{}%
{\itshape}{}%
{\bfseries}{}%  % Note that final punctuation is omitted.
{\newline}{}

\newtheoremstyle{sc}%
{}{}%
{}{}%
{\scshape}{}%  % Note that final punctuation is omitted.
{\newline}{}

\theoremstyle{break}
\newtheorem{thm}{Théorème}[section]
\newtheorem{lm}[thm]{Lemme}
\newtheorem{prop}[thm]{Proposition}
\newtheorem{cor}[thm]{Corollaire}

\theoremstyle{sc}
\newtheorem{exo}{Exercice}

\theoremstyle{definition}
\newtheorem{defi}[thm]{Définition}
\newtheorem{ex}[thm]{Exemple}

\theoremstyle{remark}
\newtheorem{rem}[thm]{Remarque}

% Raccourcis pour les opérateurs mathématiques (les espaces avant-après sont modifiés pour mieux rentrer dans les codes mathématiques usuels)
\DeclareMathOperator{\Ker}{Ker}
\DeclareMathOperator{\Id}{Id}
\DeclareMathOperator{\Img}{Im}
\DeclareMathOperator{\Card}{Card}
\DeclareMathOperator{\Vect}{Vect}


% Nouvelles commandes
\newcommand{\ps}[2]{\left\langle#1,#2\right\rangle}
\newcommand{\ent}[2]{[\![#1,#2]\!]}
\newcommand{\diff}{\mathop{}\!\mathrm{d}}

% opening
%\title{}
%\author{}



\begin{document}

%\maketitle

%\begin{abstract}

%\end{abstract}

%\tableofcontents

%\clearpage

\begin{exo}
  Soit $f:[0,+\infty[\rightarrow\mathbb{R}$ une fonction continue admettant une
    limite $\ell$ en $+\infty$. Montrer que $\int_0^{+\infty}f(t)\diff t$ est
    divergente si $\ell\neq0$.
\end{exo}

\begin{exo}
  Démontrer que $\int_0^\infty f(t)\diff t$ est convergente. On pourra
  faire une intégration par partie.
\end{exo}

\begin{exo}
  Soit $\alpha\in\mathbb{R}$, déterminer une condition nécessaire et
  suffisante sur $\alpha$ pour que $\int_0^\infty
  t^{\alpha-1}e^{-t}\diff t$ soit convergente.
\end{exo}

\begin{exo}
  Soit $f:[0,1]\rightarrow\mathbb{R}$ une fonction continue.
  \begin{enumerate}
    \item Montrer que $\int_0^1\frac{f(x)}{\sqrt x}\diff x$ est
      convergente. On pourra utiliser le théorème sur l'image d'un
      segment par une fonction continue.
    \item Montrer que si $f$ est strictement positive,
      $\int_0^1\frac{f(x)}{x}\diff x$ est divergente.
    \item Établir la même conclusion en supposant seulement que
      $f(0)>0$.
  \end{enumerate}
\end{exo}

\begin{exo}
  Pour $x>0$, on note $\Gamma(x)=\int_0^\infty t^{x-1}e^{-t}\diff t$.
  \begin{enumerate}
    \item Montrer que cette intégrale est bien définie pour tout $x>0$.
    \item Justifier $\forall x>1,\; \Gamma(x)=(x-1)\Gamma(x-1)$ et calculer
      $\Gamma(n)$ pour $n\in\mathbb{N}^*$.
  \end{enumerate}
\end{exo}

\begin{exo}
  Calculer les intégrales suivantes :
  \[
    \int_0^\infty \frac{\diff
    t}{(t+1)(t+2)}\phantom{ainsi que}\int_0^\infty\frac{\diff t}{(e^t+1)(e^{-t}+1)}
  \]
  \[
    \int_0^\infty\ln(1+\frac{1}{t^2})\diff t\phantom{ainsi
    que}\int_0^\infty\exp(-\sqrt t)\diff t
  \]
  \[
    \int_0^\infty \frac{\diff t}{\sqrt{e^t+1}}\phantom{ainsi
    que}\int_0^\infty\frac{\diff t}{\mathrm{sh}(t)}
  \]
  \[
    \int_0^1\frac{\ln(t)}{\sqrt t}\diff t
  \]
\end{exo}

\begin{exo}
  Soit $f:[0,+\infty[\rightarrow\mathbb{R}$ de classe $\mathcal C^1$
    telle que $f$ et $f'$ soient intégrables sur $[0,+\infty[$. Montrer
      que $f$ tend vers $0$ en $+\infty$.
\end{exo}

\begin{exo}
  Soit $f:[0,+\infty[$ une fonction continue par morceaux. On suppose
    que $\int_0^\infty f(t)\diff t$ est convergente. Montrer que
    $\int_x^{x+1}f(t)\diff
    t\;\underset{x\rightarrow+\infty}{\longrightarrow}0$
\end{exo}

\begin{exo}
  Montrer l'existence et déterminer la valeur de
  $I(a)=\int_0^\infty\sin(t)e^{-at}.$
\end{exo}

\begin{exo}
  Soit $f$ définie par $f(a)=\int_1^\infty \frac{\diff t}{t^a+1}$.
  \begin{enumerate}
    \item Pour quelles valeurs de $a$ $f$ est-elle bien définie ?
    \item Montrer que $f$ est décroissante de limite nulle.
  \end{enumerate}
\end{exo}

\end{document}
